\chapter*{Abstract}
\label{cha:abtract} 

Spatial indexes like R-Trees degrade in performance when the data partitions are unbalanced and their bounding boxes overlap significantly.
This thesis presents a capacity-aware clustering pipeline for index partitioning, which starts with a standard clustering algorithm like $k$-Means and then optimizes the clusters with a post-processing algorithm to minimize overlap and balance their sizes.
Oversized clusters are recursively split by KD-style median cuts along the longest axis, while undersized clusters are greedily merged into the neighbor that minimally increases the union MBR volume.
This enforces hard constraints on the cluster sizes while minimizing overlap.
The optimized clusters are then used as partitions for building an R-Tree, which improves query performance significantly.
\newline \newline
An extensive experimental evaluation on synthetic datasets in different dimensions and data distributions shows that the optimization achieves a median reduction of $\approx 8.3\%$ and a mean reduction of $\approx 7.7\%$ in node visits, with positive effects in roughly 60–65\% of scenarios.
The largest gains occur with linear split heuristics.
% \paragraph{Keywords} A, B, C

