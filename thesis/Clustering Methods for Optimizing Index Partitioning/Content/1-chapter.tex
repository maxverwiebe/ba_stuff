
\chapter{Introduction}
The first chapter introduces the motivation, problem statement, goals, and contributions of this thesis.
It also provides an overview of the structure of the thesis.
\section{Motivation}
With the continuous growth of digital data sets, the efficient organization of data becomes more and more important. 
Especially for spatial data, the underlying data structure is crucial when it comes to performance.
Spatial data refers to multi-dimensional data. For example, data in geographic information systems, navigation applications, or machine learning use cases.
Spatial index structures, like R-Trees, are widely used for that, but they reach their limits when dealing with large amounts of data and uneven data distributions.
Overlapping and unbalanced partitions mean that queries have to pass through an unnecessary number of nodes in an R-Tree index, which increases both runtime and I/O costs.
\newline\newline
A promising way to avoid these problems is a form of preprocessing the data before indexing it.
We can reach this through clustering to create partitions that are more compact, balanced, and better separated from each other.

\section{Problem Statement}
The main problem with R-Trees lies in the way data is inserted into the index.
The R-Tree heuristic often produces large, uneven data partitions with big spatial overlaps. These overlaps lead to inefficiencies during querying, as multiple branches of the tree must be traversed.
\newline\newline
The aim of this thesis is therefore to develop a method that creates more optimized partitionings in R-Trees.
\newline\newline
The central question is: How can clustering methods such as $k$-Means be used and extended to achieve more efficient partitioning for R-Trees?

\section{Goals and Contributions}
The main objective of this thesis is the development of an algorithm to optimize clusters regarding size constraints and overlap of bounding boxes. 
Whereas the main contributions of this work are:
\begin{itemize}
  \item The combination of a $k$-Means approach with a subsequent optimization process.
  \item The integration of this process into the R-Tree index building process.
  \item A comprehensive experimental evaluation based on 1080 test cases.
\end{itemize}

\section{Structure of the Thesis}
The structure of this thesis is as follows:
Chapter 2 introduces the basic concepts of this thesis, including data, clustering, bounding boxes, and R-Trees.
Chapter 3 provides an overview of related work. Chapter 4 describes the problem context.
Chapters 5 and 6 first explain baseline clustering with $k$-Means and then the developed optimization algorithm to perform post-processing.
Chapter 7 shows the integration of the approach into the structure of R-Trees.
Chapter 8 presents the experiments and their results. Finally, Chapter 9 summarizes the key findings and provides an outlook on possible extensions.
